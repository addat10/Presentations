%%%%%%%%%%%%%%%%%%%%%%%%%%%%%%%%%%%%%%%%%%%%%%%%%%%%%%%%%%%%%%%%%%%%%
%%%%%%%%%%%%%%%%%%%%%%%%%%%%%%%%%%%%%%%%%%%%%%%%%%%%%%%%%%%%%%%%%%%%%
\begin{frame}{Coupled architecture: Consensus/ Formation control}	
	\begin{itemize}
		\item Block diagram
		\item Some related work
		\begin{itemize}
			\item Consensus(Formation) with a fixed Laplacian [Fax and Murray,??] -> Modal decomposition
			\item Consensus(Formation) with a uncertain Laplacian [Popov, eichler, Hoffman,??] -> Modal decomposition
			\item Flocking with a fixed Laplacian [Francis]	
			\item Interconnections of dissipative systems -> Mark Spong		
		\end{itemize}
	\end{itemize}
\end{frame}
%%%%%%%%%%%%%%%%%%%%%%%%%%%%%%%%%%%%%%%%%%%%%%%%%%%%%%%%%%%%%%%%%%%%%
%%%%%%%%%%%%%%%%%%%%%%%%%%%%%%%%%%%%%%%%%%%%%%%%%%%%%%%%%%%%%%%%%%%%%
\begin{frame}{Coupled architecture: Flocking}	
	\begin{itemize}
		\item Flocking of double integraor agents \footnote{Rantzer,2011}
		\item Flocking of robotic agents \footnote{Olfati Saber,2011}
		\item Open problem
	\end{itemize}
\end{frame}
%%%%%%%%%%%%%%%%%%%%%%%%%%%%%%%%%%%%%%%%%%%%%%%%%%%%%%%%%%%%%%%%%%%%%
%%%%%%%%%%%%%%%%%%%%%%%%%%%%%%%%%%%%%%%%%%%%%%%%%%%%%%%%%%%%%%%%%%%%%
\begin{frame}{Coupled architecture: Flocking and source-seeking}	
	\begin{itemize}
		\item Block diagram with double integrators
		\item Analysis result including the external field
		\item Experimental results
	\end{itemize}
\end{frame}
%%%%%%%%%%%%%%%%%%%%%%%%%%%%%%%%%%%%%%%%%%%%%%%%%%%%%%%%%%%%%%%%%%%%%
%%%%%%%%%%%%%%%%%%%%%%%%%%%%%%%%%%%%%%%%%%%%%%%%%%%%%%%%%%%%%%%%%%%%%
\begin{frame}{Coupled architecture: Flocking and source-seeking with non-linear agents}	
	\begin{itemize}
		\item Block diagram with double integrators and non-linear agent
		\item Analysis result including the external field
		\item Simulation results
	\end{itemize}
\end{frame}
%%%%%%%%%%%%%%%%%%%%%%%%%%%%%%%%%%%%%%%%%%%%%%%%%%%%%%%%%%%%%%%%%%%%%
%%%%%%%%%%%%%%%%%%%%%%%%%%%%%%%%%%%%%%%%%%%%%%%%%%%%%%%%%%%%%%%%%%%%%
\begin{frame}{Coupled architecture: Some speculative ideas and open questions}	
	\begin{itemize}
		\item Use IQCs to obtain exponential convergence rates of local closed loops and use  singular perturbation argument such as in [Mesbahi] 
	\end{itemize}
\end{frame}